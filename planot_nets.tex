\documentclass[12pt, a4paper, openany]{amsart}
% \usepackage{amsfonts,amsmath,amssymb,amsthm}
% \usepackage{array,longtable,hhline,mathabx}
\usepackage[english,russian]{babel}
\usepackage[utf8]{inputenc}

\usepackage{comment}

\usepackage{xcolor}
\usepackage{bbm}

% \usepackage[all]{xypic}
% \xyoption{rotate}
% \xyoption{pdf}


\theoremstyle{plain}
\newtheorem{theorem}{Теорема}
\newtheorem{cor}[theorem]{Следствие}
\newtheorem{prop}[theorem]{Предложение}
\newtheorem{lemma}[theorem]{Лемма}
\newtheorem{statement}[theorem]{Statement}
\newtheorem{sbl}[theorem]{Sublemma}
\newtheorem{conj}[theorem]{Conjecture}
\newtheorem{pr}[theorem]{Property}
\newtheorem{ax}[theorem]{Axiom}

\theoremstyle{definition}
\newtheorem{de}[theorem]{Определение}

\theoremstyle{remark}
\newtheorem{rem}[theorem]{Замечание}
\newtheorem{ex}[theorem]{Упражнение}
\newtheorem{e}[theorem]{Пример}
\newtheorem{que}[theorem]{Вопрос}

\newtheorem{example}[theorem]{Пример}
\newtheorem{hint}[theorem]{Указание}


\newcommand{\slava}[1]{{\color{blue}#1}}
\newcommand{\slavacom}[1]{{\color{blue}\it#1}}
\def\slavapar#1\par{\slava{#1}\par}

\begin{document}

{\color{green}Картинка в начале статьи: бородатый мужик в
  "древнегреческом" наряде тащит из моря сеть, в которой видны разные
  топологические многообразия.  (В стилистике старого "Кванта" или
  журнала "ХиЖ".)}
	
	
{\color{green} В начале статьи следует рассказать о следующих темах:}
	
	
{\color{green}Платоновы тела. Двойственность. Символ Шлефли.}
		
{\color{green}Ориентируемые поверхности. Тор, крендель и т.д. ...}
		
{\color{green}Графом мы будем называть ...}
		
{\color{green}Формула Эйлера для графа на поверхности ...}
		
{\color{green} А что можно сказать про неориентируемый случай,
  допустим, бутылку Кляйна??}


\slavacom{Может лучше ``вложен в поверхность''? Это стандартный термин.}
Граф правильно нарисован на поверхности, если его ребра не
пересекаются вне вершин и являются гладкими кривыми на поверхности
(что бы это ни означало). Связный граф, правильно нарисованный на
некоторой поверхности мы будем называть сетью на этой поверхности,
если он не имеет вершин степени один.
\slavacom{Можно не плодить лишних сущностей, потом ведь есть только
  платоновы графы}


Посредством своих ребер сеть разделяет поверхность на непересекающиеся
области.  Такие области мы будем называть странами. Предположим, что
\begin{itemize}
\item Каждое ребро разделяет две различные страны;
  \slavacom{Это где-то используется?}
\item В каждой вершине сходятся попарно различные страны; 
  \slavacom{Это где-то используется?}
\item Каждая страна (топологически эквивалентна) гомеоморфна
  многоугольнику, чья граница представлена циклом из вершин и ребер.
\item Граф не имеет мультиребер (ниже будет рассмотрено исключение ---
  случай разбиения сферы по схеме "мандарина"). Следует, вероятно,
  также потребовать выполнения двойственного условия --- никакие две
  страны не граничат через два различных ребра.
   \slavacom{Зачем, если всё рабно рассматриваем мандарин?}
\end{itemize} 

Такую сеть будем называть картой.
	
Наконец, карту назовем платоновой $(k,n)$-картой, если в каждой
вершине сходится одинаковое количество ребер --- $n$ и каждая страна
эквивалентна $k$-угольнику.

\slavapar
Будем говорить, что граф вложен в некоторую поверхность, если он
нарисован на поверхности таким образом, что все точки
(само-)пересечений рёбер это его вершины. Для пары натуральных чисел
$k,n\geq2$, платоновой $(k,n)$-картой назовём такой
вложенный в поверхность граф, что
\begin{itemize}
\item Валентность каждой вершины (количество подходящих к вершине
  концов рёбер) равна $k$.
\item Каждая \textit{страна} (одна из связных областей, на которые наш
  граф разбивает поверхность) является топологическим
  $n$-угольником, то есть топологическим диском ограниченным ровно $n$
  рёбрами.  
\end{itemize} 

\slavapar
Заметим, что рёбро может а) уткнуться двумя своими концами в одну и ту
же вершину; б) участвовать в границе одной страны дважды, если страна
подходит к ребру с двух разных сторон.  Карты, для которых хотя бы одно
из этих условий выполняется для какого-либо ребра, назовём \textit{особыми}, а
другие карты \textit{неособыми}.

\slavapar
Для каждой платоновой $(k,n)$-карты, можно построить двойственную ей
$(n,k)$-карту следующим образом. Выберем по точке внутри каждой страны
(столицу) и для каждого ребра проведём ребро двойственной карты
пересекающее данное в одной точке и соединяющее столицы стран
соседствующих через данное ребро. Несложно увидеть, что при переходе к
двойственной карте (не)особость карты сохраняется, а условия а) и б)
выше меняются местами.



\slavacom{Мне кажется, лучше не использовать русские буквы, выглядящие
как латинские. Те Щ,Л -- нормально, а А,В,С -- плохо.}	
Заметим теперь некоторые численные соотношения, которые выполнены для
всякой платоновой карты. Обозначим числа вершин, ребер и граней
(стран) на ней через $B,P, и $ $\Gamma$, соответственно.  Тогда,
поскольку в каждой вершине сходится ровно $k$ ребер, и каждое ребро
соединяет две различных вершины, имеем: $2P=kB$. Похожим образом,
каждая страна ограничена $n$ ребрами, и каждое ребро разделяет две
страны, откуда мы получаем $2P=n\Gamma$. Попробуем теперь
получить ограничения на платоновы карты при помощи формулы Эйлера
$$
B-P+\Gamma=2(1-g).
$$
Добавив полученные соотношения и сделав подстановку в формуле Эйлера
получаем систему диофантовых уравнений
\begin{equation}\label{eq:eiler}
  \begin{cases}
    \frac{1}{k}+\frac{1}{n}+\frac{g-1}{P}=\frac{1}{2}\\
    B=\frac{2P}{k}\\
    \Gamma=\frac{2P}{n}
  \end{cases}
\end{equation}
которую необходимо решить при условиях, $k,n\geq2$ и $B,P,\Gamma\geq1$.

Рассмотрим вначале случай $g=0$.  Неизвестные $k$ и $n$ входят в
формулу симметрично. следовательно, мы можем предположить, что
$k\ge n$. Случай $n=2$ мы рассмотрим отдельно. Итак, будем считать,
что $3\le n\le k$.  Заметим, что ситуация $k>5$
невозможна. Действительно, значение
$\frac{1}{k}+\frac{1}{n}-\frac{1}{2}$, очевидно, должно быть
положительным.  Но мы получаем:
$$
\frac{1}{k}+\frac{1}{n}-\frac{1}{2}\le \frac{1}{6}+\frac{1}{3}-\frac{1}{2}=0.
$$
По аналогичным причинам ($\frac{1}{4}+\frac{1}{4}-\frac{1}{2}=0$),
невозможна ситуация $n>3$.  Таким образом, нам надо рассмотреть лишь
случаи $(3,3), (3,4), (3,5)$.  С учетом симметрии $k$ и $n$, надо
добавить еще пары $(4,3)$ и $(5,3)$.  Эти пять случаев и соответствуют
пяти правильным многогранникам --- платоновым телам!
	   
Прежде чем перейти к поверхностям рода $g>0$, рассмотрим отложенные
нами случаи, когда $k$ или $n$ равны двум. Эти два случая, конечно,
двойственны друг другу. Предположим, что $n=2$. В этом случае наша
формула превращается в соотношение $P=k$ и мы рассматриваем разбиение
сферы на двуугольники.  Ясно, что такие разбиения существуют для
каждого $k>1$, их возможная реализация, напоминающая разделенный на
дольки мандарин, выглядит следующим образом. Возьмем две
противоположные точки сферы --- полюса, и соединим их $k$ ребрами ---
меридианами. В результате мы получим разбиение сферы типа
$(k,2)$. Двойственный случай $(2,n)$ получится, если провести на сфере
экватор, отметить на нем $n$ вершин. Две появившиеся полусферы будут
играть роль $n$-угольных граней. Все полученные платоновы карты на
сфере являются неособыми.
	   
Рассмотрим теперь случай $g=1$. Это даст нам ограничения на платоновы
карты на торе. В этом случае слагаемое $\frac{g-1}{P}$ нашего
уравнения~\ref{eq:eiler} обращается в нуль.  Соответственно, мы не получим
никаких условий на числа ребер в нашей платоновой сети. Но необходимым
условием существования такой сети будет равенство
$$
\frac{1}{k}+\frac{1}{n}=\frac{1}{2}.
$$
Несложно установить, все решения этого диофантова уравнения:
$(3,6), (4,4), (6,3)$. Следовательно, платоновы сети на торе могут
состоять либо из четырехугольников, сходящихся по четыре в каждой
вершине, либо из треугольников, сходящихся по шесть, либо, наконец, из
шестиугольников, сходящихся по три. Последние две сети двойственны
друг другу.

\begin{e}
  Здесь примеры (картинки) разбиения на четырехугольники,
  шестиугольники (кирпичная кладка), треугольники (режем все
  прямоугольники по одной диагонали).
\end{e}	   
	   
\begin{ex}
  Можно ли построить на торе неособую платонову сеть типа $(4,4)$,
  имеющцю простое число стран?
\end{ex}	     


Теперь перейдем к случаю $g\ge 2$. 

При $g=2$, пользуясь~\ref{eq:eiler}, мы получим симметричное
диофантово уравнение
\begin{equation}\label{eq:eilerg2}
  \begin{cases}
    \frac{1}{k}+\frac{1}{n}+\frac{1}{P}=\frac{1}{2}\\
    B=\frac{2P}{k}\\
    \Gamma=\frac{2P}{n}
  \end{cases}
\end{equation}
Выпишем все решения нашего уравнения при условии $k\leq n$, остальные
решения получаются при одновременной замене $k\leftrightarrow n$ и
$B\leftrightarrow\Gamma$.

\begin{tabular}{c|cccccccccccccc}
  $k$ & 3 & 3 & 3 & 3 & 3 & 3 & 4 & 4 & 4 & 4 & 5 & 5 & 6 & 8 \\
  $n$ & 7 & 8 & 9 & 10 & 12 & 18 & 5 & 6 & 8 & 12 & 5 & 10 & 6 & 8 \\
  \hline
  $B$ & 28 & 16 & 12 & 10 & 8 & 6 & 10 & 6 & 4 & 3 & 4 & 2 & 2 & 1\\
  $E$ & 42 & 24 & 18 & 15 & 12 & 9 & 20 & 12 & 8 & 6 & 10 & 5 & 6 & 4 \\
  $\Gamma$ & 12 & 6 & 4 & 3 & 2 & 1 & 8 & 4 & 2 & 1 & 4 & 1 & 2 & 1\\
\end{tabular}
                                                                
Неособость карты влечёт за собой дополнительные неравенства $B>$


Отнюдь не все из этих решений могут соответствовать платоновым картам.
Попробуем разобраться с этим вопросом, установив, какие еще неравенства связывают
числа, характеризующие платонову сеть. Для этого мы добавили к нашей таблице числа вершин и граней (стран), которые должна иметь платонова карта типа $(k,n)$. Напомним, что они вычисляются по формулам $B=2P/k$, $\Gamma=2P/n$.



Из условий, определяющих платонову сеть, ясно, что общее количество вершин должно быть больше $k$, а общее количество стран --- больше $n$. Поэтому, у нас остаются лишь следующие потенциальные решения:


\begin{tabular}{ccc|cc}
	k&n&P&B&$\Gamma$\\	
	3&7&42&28&12\\
	4&5&20&10&8\\			
\end{tabular}

а также двойственные к ним $(7,3,42)$ и $(5,4,20)$.

\begin{que}
	Отвечает ли в реальности какое-нибудь из этих решений платоновой карте на кренделе, или таких платоновых карт не существует?? (Ответ мне не известен.) 
\end{que}

Заметим, что из полученных нами целочисленных решений для $g=2$, сразу же получаются некоторые решения для случая $g>2$. Например, найденная нами тройка
$(3,7,42)$ даст нам общее решение $(3,7,42(g-1))$. 

В то же время, для $g>2$, стоит ожидать появления и других целочисленных решений,
которые соответствуют рациональным, но не целым решениям уравнения~\ref{two}. 

 
 
	
%	Домножая на $2kn$ и упрощая, получим:
%	$$
%	P(2n-kn+2k)=2kn(1-g).
%	$$
%	Рассмотрим вначале случай $g=0$.
%	Формула будет выглядеть так:
%	$$
%	P(2n-kn+2k)=2kn.
%	$$
	
	
\begin{comment}	
	
	
	
	\hrulefill
	
	Вспомним о правильных многогранниках.  (тетраэдр, куб(гексаэдр), октаэдр, икосаэдр, додекаэдр).
	Попробуем доказать, что других не существует. Причем мы будем пользоваться не геометрическими, а топологическими соображениями.
	Для этого мы будем, вместо многогранников, рассматривать нарисованные на поверхностях графы.
	
	Нам понадобится  следующее определение.
	Для нас граф --- это конечное множество вершин $\{v_1,\ldots, v_n\}$, некоторые из которых объединены в пары (соединены ребрами) $e_{ij}=(v_i,v_j)$. Если не оговорено обратное, мы будем предполагать, что
	каждой паре вершин $ij$ соответствует не более одного ребра.
	Ясно, что графам соответствует известный способ их изображения в виде
	точек соединенных кривыми.
	
	
	
	Мы будем говорить, что граф правильно нарисован на поверхности, если изображения его ребер  на этой поверхности пересекаются только в вершинах.
	Также, назовем нарисованный на замкнутой поверхности граф сетью, если он связен, не имеет вершин степени один, и разбивает поверхность на области, гомеоморфные многоугольникам. Такие области на поверхности будем называть гранями, подчеркивая сходство наших графов с многогранниками.
	Основное свойство, которое характеризует правильные многогранники, то, что 
	их грани имеют одинаковое число вершин, и в каждой вершине сходится одинаковое число граней (или, что то же, ребер). Аналогичное свойство легко
	определить и на сетях. Назовем сеть платоновой, если все ее вершины имеют одинаковую степень $k$, а все грани, с учетом ограничивающих их вершин, эквивалентны (гомеоморфны) $n$-угольникам.  Таким образом, мы можем охарактеризовать правильные многогранники их парами $(k,n)$. 
	Тетраэдр $(3,3)$, куб $(3,4)$, октаэдр $(4,3)$, икосаэдр $(5,3)$, додекаэдр $(3,5)$. В дальнейшем, естественно предполагать, что $k\ge 2$ $n\ge 3$.
	
	Применение формулы Эйлера к сетям на поверхности позволяет нам получить
	ограничения на возможные типы сетей.
	
	$$
	B-P+\Gamma=\chi=2-2g
	$$
	
	Заметим, что $B=\frac{n\Gamma}{k}$, а $P=\frac{n\Gamma}{2}$.
	Подставим эти значения в формулу Эйлера и избавимся от знаменателей.  Получим:
	$$
	\Gamma(2(k+n)-kn)=2k\chi.
	$$
	
	Это выражение симметрично относительно вхождения $k$ и $n$. Поэтому,
	мы можем написать его для двойственной сети, получив 
	$$
	B(2(k+n)-kn)=2n\chi.
	$$
	
	Таким образом, $2(k+n)-kn$ делит наибольший общий делитель $2k\chi$
	и $2n\chi$. Но $2(k+n)-kn$ само делится на $(n,k)$. Значит,
	$\frac{2(k+n)-kn}{(n,k)}$ делит $2k\chi$.
	Мы получилим необходимое условие существования платоновой сети типа $(n,k)$
	на поверхности эйлеровой характеристики $\chi$.
	
	Будем называть значение $2(k+n)-kn$ индексом платоновой сети типа $(k,n)$.
	Очевидно, $ind(2,k)=3-(-1)^k$.
	
	Приведем таблицу индексов для небольших значений $n>2$ и $k>2$.
	
	 \begin{table}[]
		\begin{tabular}{l|llllllllll}
			& 3 & 4 & 5 & 6 & 7 & 8 & 9 & 10 &11 &12\\
			\hline
			3& 1 & 2 & 1 & 0 & -1 & -2 & -1 & -4 & -5& -2\\
			4&  & 0 & -2 & -2  & -6 & -2 & -10 & -6 & -14& -4\\
			5&  &  & -1 & -8  & -11 & -14 & -17 & -4 & -23& -26\\
			6&  &  &  & -2  & -16 & -10 & -8 & -15 & -32& -6\\
			7&  &  &  &   & -3 & -26 & -31 & -36 & -41& -46\\
			8&  &  &  &   & & -4 & -38 & -22 & -50& -14\\
			9&  &  &  &   & & & -5 & -52 & -59& -22\\
			10& & & & &&&&-6& -68& -38\\
			11&&&&&&&&&-7&-86\\
			12&&&&&&&&&&-8
		\end{tabular}
	\end{table}
	 
	
	 Применим этот результат  к классификации платоновых сетей на сфере.
%	 В этом случае $\chi=2$, значит индекс сети должен делить $4k$
	 
	    
	
	\end{comment}
	    
 
		 
\end{document}